%*----------- SLIDE -------------------------------------------------------------
\begin{frame}[c]{Introdução} 
    \transdissolve[duration=0.5]
   
    \begin{center}
        \Wider{%
        \begin{shaded}
        \begin{center}
            \vspace*{0.5cm}
            \resizebox{!}{0.7cm}{%
                \color{bg} O que é o raspode?
            }%
        \end{center}
        \end{shaded}
        }%
    \end{center}
    
\end{frame}

\begin{frame}[t]{Introdução} 
    \transdissolve[duration=0.5]
    Robô hexapode que tem como objetivo se tornar uma plataforma de aprendizagem para navegação autônoma e cinemática inversa.
    %\newline
        \begin{columns}[t]
            \column{.05\linewidth}
            \column{.5\linewidth}
                \begin{itemize}
                    \item IEEE RAS CIMATEC;
                    \item ROS2 Foxy;
                    \item Navegação autônoma;
                    \item Mapeamento de ambientes.
                \end{itemize}
            \column{.4\linewidth}
            \begin{center}
            %\centerline{
                \begin{figure}
                    \includegraphics[width=0.8\textwidth]{cad.png}
                    \caption{Modelo 3D(Voser, 2021).}
                    % \roundpic[xshift=0cm,yshift=0cm]{2.5cm}{6cm}{pista}
                    %\caption{Pista de corrida \cite{agostini2007}}
                \end{figure}
            %}
            \end{center}
        \end{columns}
%*----------- notes
    \note[item]{Notes can help you to remember important information. Turn on the notes option.}
\end{frame}

\begin{frame}[t]{Objetivos}
    Desenvolver competência em:
    \begin{itemize}
        \item ROS 2 - Foxy;
        \item Cinemática inversa;
        \item Sistemas de controle aplicado a patas de 3 juntas;
        \item Navegação Autônoma utilizando planejadores de trajetória;
        \item Mapeamento 3D de ambientes utilizando visão computacional.
    \end{itemize}
%*----------- notes
    \note[item]{Notes can help you to remember important information. Turn on the notes option.}
\end{frame}

\begin{frame}[t]{Metodologia} 
    \framesubtitle{DIP\&T}
    \transdissolve[duration=0.5]
    \centering
    \begin{figure}
        
        \includegraphics[width=0.8\textwidth]{dipt.png}\\
        \caption{Etapas do Processo utilizando o DIP\&T}
    \end{figure}
    
\end{frame}

\begin{frame}[t]{Metodologia} 
    \framesubtitle{Adaptação do DIP\&T}
    \transdissolve[duration=0.5]
    \centering
    \begin{figure}
        \includegraphics[width=.95\textwidth]{dipt-setps.png}\\
        \caption{Etapas da metodologia.}
    \end{figure}
    
\end{frame}


\begin{frame}[t]{Metodologia} 
    \transdissolve[duration=0.5]
    \centering
    \includegraphics[width=\textwidth]{diagram-4.pdf}
\end{frame}



\begin{frame}[t]{Inspiração}
    \framesubtitle{Leptan project}

    \transdissolve[duration=0.5]
    \centering
    \includegraphics[width=0.5\textwidth]{real.png}

\end{frame}

\begin{frame}[t]{Inspiração}
    \transboxout[duration=0.5]
    \framesubtitle{Leptan project}
    \begin{columns}
        \column{.1\textwidth}
        \column{.4\textwidth}
            \includegraphics[width=.87\textwidth]{real.png}
        \column{.6\textwidth}
            Através de teleoperação, é possível:
            \begin{itemize}
                \item Levantar e deitar;
                \item Caminhar em qualquer direção;
                \item Girar em torno do seu próprio eixo.
            
            \end{itemize}
    \end{columns}
   
\end{frame}

\begin{frame}[t]{Modelo 3D}

    \transdissolve[duration=0.5]
    \centering
    \includegraphics[width=0.7\textwidth]{cad}

\end{frame}

%*----------- SLIDE -------------------------------------------------------------
\begin{frame}[c]{Simulação}
    \framesubtitle{Estado Atual - Controlado por Teleoperação}
    \begin{center}
    
        \includemedia[
            width=0.7\linewidth,
            totalheight=0.39375\linewidth,
            activate=pageopen,
            passcontext, 
            %transparent,
            addresource=./Source/movies/raspode-movement.mp4,
            flashvars={
            source=./Source/movies/raspode-movement.mp4
            &autoPlay=true
            &autoRewind=true
            &loop=true}
            ]{\fbox{\includegraphics{Source/pictures/cad.png}}}{VPlayer.swf}
    \end{center}

%*----------- notes
    \note[item]{Notes can help you to remember important information. Turn on the notes option.}
 \end{frame}
%-
%*----------- SLIDE -------------------------------------------------------------
